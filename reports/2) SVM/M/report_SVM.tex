\documentclass[a4paper,12pt]{article}

\usepackage[utf8x]{inputenc}
\usepackage[english, russian]{babel}

\usepackage{tabularx}
\usepackage{multirow}
\usepackage{graphicx}
\usepackage{longtable}
\usepackage{misccorr}
\usepackage{indentfirst}
\usepackage{amsmath}
%\usepackage{fancynum}


\usepackage{listings}
\usepackage{xcolor}

\usepackage{fullpage}

\usepackage[labelsep=endash,
		    margin=10pt, 
		    justification = centerlast, 
		    format = hang,
		    singlelinecheck=false
		    ]{caption}

\exhyphenpenalty=10000
\doublehyphendemerits=10000
\finalhyphendemerits=5000

\definecolor{codegreen}{rgb}{0,0.6,0}
\definecolor{codegray}{rgb}{0.5,0.5,0.5}
\definecolor{codepurple}{rgb}{0.58,0,0.82}
\definecolor{backcolour}{rgb}{0.95,0.95,0.92}

\newcommand{\tracking}[2]{#2}
\input{letterspacing.tex}\renewcommand{\tracking}[2]{\mbox{\letterspace to #1\naturalwidth{#2}}}
 
\lstdefinestyle{mystyle}{
    backgroundcolor=\color{backcolour},
    commentstyle=\color{codegreen},
    keywordstyle=\color{blue},
    numberstyle=\tiny\color{codegray},
    stringstyle=\color{codepurple},
    basicstyle=\footnotesize,
    breakatwhitespace=false,
    breaklines=true,
    captionpos=t,
    keepspaces=true,
    numbers=left,
    numbersep=10pt,
    showspaces=false,
    showstringspaces=false
    showtabs=false,
    tabsize=4,
    frame=tb
}
 
\lstset{style=mystyle}

\usepackage{color}
\usepackage{xcolor}
\usepackage{listings}
 
% Цвета для кода
 
\definecolor{string}{HTML}{B40000} % цвет строк в коде
\definecolor{comment}{HTML}{008000} % цвет комментариев в коде
\definecolor{keyword}{HTML}{1A00FF} % цвет ключевых слов в коде
\definecolor{morecomment}{HTML}{8000FF} % цвет include и других элементов в коде
\definecolor{сaptiontext}{HTML}{FFFFFF} % цвет текста заголовка в коде
\definecolor{сaptionbk}{HTML}{999999} % цвет фона заголовка в коде
\definecolor{bk}{HTML}{FFFFFF} % цвет фона в коде
\definecolor{frame}{HTML}{999999} % цвет рамки в коде
\definecolor{brackets}{HTML}{B40000} % цвет скобок в коде
 

%%% Отображение кода %%%
 
% Настройки отображения кода
 
\lstset{
	%morekeywords={*,...}, % если хотите добавить ключевые слова, то добавляйте	 
	% Настройки отображения     
	breaklines=false, % Перенос длинных строк
	% Для отображения русского языка
	extendedchars=true,
	literate={Ö}{{\"O}}1
	{Ä}{{\"A}}1
	{Ü}{{\"U}}1
	{ß}{{\ss}}1
	{ü}{{\"u}}1
	{ä}{{\"a}}1
	{ö}{{\"o}}1
	{~}{{\textasciitilde}}1
	{а}{{\selectfont\char224}}1
	{б}{{\selectfont\char225}}1
	{в}{{\selectfont\char226}}1
	{г}{{\selectfont\char227}}1
	{д}{{\selectfont\char228}}1
	{е}{{\selectfont\char229}}1
	{ё}{{\"e}}1
	{ж}{{\selectfont\char230}}1
	{з}{{\selectfont\char231}}1
	{и}{{\selectfont\char232}}1
	{й}{{\selectfont\char233}}1
	{к}{{\selectfont\char234}}1
	{л}{{\selectfont\char235}}1
	{м}{{\selectfont\char236}}1
	{н}{{\selectfont\char237}}1
	{о}{{\selectfont\char238}}1
	{п}{{\selectfont\char239}}1
	{р}{{\selectfont\char240}}1
	{с}{{\selectfont\char241}}1
	{т}{{\selectfont\char242}}1
	{у}{{\selectfont\char243}}1
	{ф}{{\selectfont\char244}}1
	{х}{{\selectfont\char245}}1
	{ц}{{\selectfont\char246}}1
	{ч}{{\selectfont\char247}}1
	{ш}{{\selectfont\char248}}1
	{щ}{{\selectfont\char249}}1
	{ъ}{{\selectfont\char250}}1
	{ы}{{\selectfont\char251}}1
	{ь}{{\selectfont\char252}}1
	{э}{{\selectfont\char253}}1
	{ю}{{\selectfont\char254}}1
	{я}{{\selectfont\char255}}1
	{А}{{\selectfont\char192}}1
	{Б}{{\selectfont\char193}}1
	{В}{{\selectfont\char194}}1
	{Г}{{\selectfont\char195}}1
	{Д}{{\selectfont\char196}}1
	{Е}{{\selectfont\char197}}1
	{Ё}{{\"E}}1
	{Ж}{{\selectfont\char198}}1
	{З}{{\selectfont\char199}}1
	{И}{{\selectfont\char200}}1
	{Й}{{\selectfont\char201}}1
	{К}{{\selectfont\char202}}1
	{Л}{{\selectfont\char203}}1
	{М}{{\selectfont\char204}}1
	{Н}{{\selectfont\char205}}1
	{О}{{\selectfont\char206}}1
	{П}{{\selectfont\char207}}1
	{Р}{{\selectfont\char208}}1
	{С}{{\selectfont\char209}}1
	{Т}{{\selectfont\char210}}1
	{У}{{\selectfont\char211}}1
	{Ф}{{\selectfont\char212}}1
	{Х}{{\selectfont\char213}}1
	{Ц}{{\selectfont\char214}}1
	{Ч}{{\selectfont\char215}}1
	{Ш}{{\selectfont\char216}}1
	{Щ}{{\selectfont\char217}}1
	{Ъ}{{\selectfont\char218}}1
	{Ы}{{\selectfont\char219}}1
	{Ь}{{\selectfont\char220}}1
	{Э}{{\selectfont\char221}}1
	{Ю}{{\selectfont\char222}}1
	{Я}{{\selectfont\char223}}1
	{і}{{\selectfont\char105}}1
	{ї}{{\selectfont\char168}}1
	{є}{{\selectfont\char185}}1
	{ґ}{{\selectfont\char160}}1
	{І}{{\selectfont\char73}}1
	{Ї}{{\selectfont\char136}}1
	{Є}{{\selectfont\char153}}1
	{Ґ}{{\selectfont\char128}}1
	{\{}{{{\color{brackets}\{}}}1 % Цвет скобок {
	{\}}{{{\color{brackets}\}}}}1 % Цвет скобок }
}

\setcounter{tocdepth}{1}

\begin{document}

\begin{titlepage}
\newpage

\

Тут титульник
\end{titlepage}

\newpage
\tableofcontents
\setcounter{page}{2}



\newpage\section{Метод опорных векторов} 
	Классификация данных - общая задача машинного обучения. Пусть некоторые заданные наблюдения (объекты) принадлежат к одному из двух классов. 

	\vspace{0.5cm}
	Задача состоит в том, чтобы определить, к какому классу будут принадлежать новые наблюдения. 

	\vspace{0.5cm}
	В случае метода опорных векторов точка в пространстве рассматривается как вектор размерности p. 

	\vspace{0.5cm}
	Вектора могут быть разделены с помощью гиперплоскостей размерности p-1. Лучшая гиперплоскость - гиперплоскость, при построении которой разделение и разница между двумя классами максимальны. 
	
	\vspace{0.5cm}
	С точки зрения точности классификации лучше всего выбрать гиперплоскость, расстояние от которой до каждого класса максимально. Другими словами, выберем ту гиперплоскость, которая разделяет классы наилучшим образом. Такая гиперплоскость, называется оптимальной разделяющей гиперплоскостью.
	
	\vspace{0.5cm}
	Вектора, лежащие ближе всех к разделяющей гиперплоскости, называются опорными векторами.
	
	
	
\newpage\section{Цель лабораторный работы} 
	Цели: 
	\vspace{0.5cm}
	
	Получить практические навыки по применению метода опорных векторов
	
	\vspace{0.5cm}
	Задачи: 
	
	\vspace{0.5cm}
	1. Применить к датасету Titanic метод опорных векторов.
	
	\vspace{0.5cm}
	2. Провести ряд экспериментов с целью получения лучших значений гиперпараметров и выбора наилучшего сочетания признаков.
	
\newpage\section{Инструменты} 
	В качестве инструментов для выполнения поставленной цели был выбран язык Python и библиотеки Skikit-learn и pandas.
	Бибилотека pandas была использована для подготовки датасета к будущему использованию.
	
	\vspace{0.5cm}
	Библиотека Skikit-learn была использована  в качестве реализации метода опорных векторов. Для этого использовалась функция SVC(Support Vector Classification).  
	
	\vspace{0.5cm}
	Основные параметры класса sklearn.svm.SVC.
	
	\vspace{0.5cm}
	C - параметр регуляризации (по умолчанию 1);
	
	\vspace{0.5cm}
	kernel - ядро классификатора. Бывает линейным, полиномиальным, сигмоидным. По умолчанию используется радикальная базисная функция;
	
	\vspace{0.5cm}
	tol - требуемая точность (по умолчанию 0.001);
	
	
	\vspace{0.5cm}
	random\_state - начальное значение для генерации случайных чисел.


	
\newpage\section{Эксперименты}
	Для того, чтобы вывести при каких параметрах функция SVC дает наиболее точное разбиение на классы, необходимо провести ряд экспериментов с разными значениями параметров. 
	
	В виду слабой вычислительной мощности машины, на которой проводится эксперимент параметр  kernel будет равен 'rfb', так как только при этом параметре реализация метода работала приемлимо быстро.
	
	В ходе экспериментов было выявлено, что  random\_state не вляиет на исход, поэтому примем его равным нулю.
	
	\vspace{0.5cm}
	Таблица 1 - Точность предстказаний в зависимости от параметров.
\begin{longtable}{|p{1cm}|p{9cm}|p{3cm}|}
\hline 
№ & Параметры & Точность (в процентах) \\ 
\hline 
1 & Параметры по умолчанию & 77.039 \\
\hline
2 & С=1; tol=0.0001; & 66.187 \\
\hline 
3 & С=1; tol=0.01; & 66.426 \\
\hline 
4 &  С=1; tol=0.5;  & 66.906 \\
\hline 
5 & С=1; tol=0.99;  & 66.666 \\
\hline
2 & С=3; tol=0.0001; & 72.182 \\
\hline 
3 & С=3; tol=0.01; & 72.182 \\
\hline 
4 &  С=3; tol=0.5;  & 71.462 \\
\hline 
5 & С=3; tol=0.99;  & 70.5 \\
\hline
2 & С=0.5; tol=0.0001; & 67.866 \\
\hline 
3 & С=0.5; tol=0.01; & 67.866 \\
\hline 
4 & С=0.5; tol=0.5;  & 66.187 \\
\hline 
5 & С=0.5; tol=0.99;  & 65.707 \\
\hline 
\end{longtable}


\newpage\section{Итог}
	В ходе проведения экпериментов были выявлены наиболее оптимальные сочетания основных параметров функции SVC из Skikit-learn. 
	
	
\end{document}
