\documentclass[a4paper,12pt]{article}

\usepackage[utf8x]{inputenc}
\usepackage[english, russian]{babel}

\usepackage{tabularx}
\usepackage{multirow}
\usepackage{graphicx}
\usepackage{longtable}
\usepackage{misccorr}
\usepackage{indentfirst}
\usepackage{amsmath}
%\usepackage{fancynum}


\usepackage{listings}
\usepackage{xcolor}

\usepackage{fullpage}

\usepackage[labelsep=endash,
		    margin=10pt, 
		    justification = centerlast, 
		    format = hang,
		    singlelinecheck=false
		    ]{caption}

\exhyphenpenalty=10000
\doublehyphendemerits=10000
\finalhyphendemerits=5000

\definecolor{codegreen}{rgb}{0,0.6,0}
\definecolor{codegray}{rgb}{0.5,0.5,0.5}
\definecolor{codepurple}{rgb}{0.58,0,0.82}
\definecolor{backcolour}{rgb}{0.95,0.95,0.92}

\newcommand{\tracking}[2]{#2}
\input{letterspacing.tex}\renewcommand{\tracking}[2]{\mbox{\letterspace to #1\naturalwidth{#2}}}
 
\lstdefinestyle{mystyle}{
    backgroundcolor=\color{backcolour},
    commentstyle=\color{codegreen},
    keywordstyle=\color{blue},
    numberstyle=\tiny\color{codegray},
    stringstyle=\color{codepurple},
    basicstyle=\footnotesize,
    breakatwhitespace=false,
    breaklines=true,
    captionpos=t,
    keepspaces=true,
    numbers=left,
    numbersep=10pt,
    showspaces=false,
    showstringspaces=false
    showtabs=false,
    tabsize=4,
    frame=tb
}
 
\lstset{style=mystyle}

\usepackage{color}
\usepackage{xcolor}
\usepackage{listings}
 
% Цвета для кода
 
\definecolor{string}{HTML}{B40000} % цвет строк в коде
\definecolor{comment}{HTML}{008000} % цвет комментариев в коде
\definecolor{keyword}{HTML}{1A00FF} % цвет ключевых слов в коде
\definecolor{morecomment}{HTML}{8000FF} % цвет include и других элементов в коде
\definecolor{сaptiontext}{HTML}{FFFFFF} % цвет текста заголовка в коде
\definecolor{сaptionbk}{HTML}{999999} % цвет фона заголовка в коде
\definecolor{bk}{HTML}{FFFFFF} % цвет фона в коде
\definecolor{frame}{HTML}{999999} % цвет рамки в коде
\definecolor{brackets}{HTML}{B40000} % цвет скобок в коде
 

%%% Отображение кода %%%
 
% Настройки отображения кода
 
\lstset{
	%morekeywords={*,...}, % если хотите добавить ключевые слова, то добавляйте	 
	% Настройки отображения     
	breaklines=false, % Перенос длинных строк
	% Для отображения русского языка
	extendedchars=true,
	literate={Ö}{{\"O}}1
	{Ä}{{\"A}}1
	{Ü}{{\"U}}1
	{ß}{{\ss}}1
	{ü}{{\"u}}1
	{ä}{{\"a}}1
	{ö}{{\"o}}1
	{~}{{\textasciitilde}}1
	{а}{{\selectfont\char224}}1
	{б}{{\selectfont\char225}}1
	{в}{{\selectfont\char226}}1
	{г}{{\selectfont\char227}}1
	{д}{{\selectfont\char228}}1
	{е}{{\selectfont\char229}}1
	{ё}{{\"e}}1
	{ж}{{\selectfont\char230}}1
	{з}{{\selectfont\char231}}1
	{и}{{\selectfont\char232}}1
	{й}{{\selectfont\char233}}1
	{к}{{\selectfont\char234}}1
	{л}{{\selectfont\char235}}1
	{м}{{\selectfont\char236}}1
	{н}{{\selectfont\char237}}1
	{о}{{\selectfont\char238}}1
	{п}{{\selectfont\char239}}1
	{р}{{\selectfont\char240}}1
	{с}{{\selectfont\char241}}1
	{т}{{\selectfont\char242}}1
	{у}{{\selectfont\char243}}1
	{ф}{{\selectfont\char244}}1
	{х}{{\selectfont\char245}}1
	{ц}{{\selectfont\char246}}1
	{ч}{{\selectfont\char247}}1
	{ш}{{\selectfont\char248}}1
	{щ}{{\selectfont\char249}}1
	{ъ}{{\selectfont\char250}}1
	{ы}{{\selectfont\char251}}1
	{ь}{{\selectfont\char252}}1
	{э}{{\selectfont\char253}}1
	{ю}{{\selectfont\char254}}1
	{я}{{\selectfont\char255}}1
	{А}{{\selectfont\char192}}1
	{Б}{{\selectfont\char193}}1
	{В}{{\selectfont\char194}}1
	{Г}{{\selectfont\char195}}1
	{Д}{{\selectfont\char196}}1
	{Е}{{\selectfont\char197}}1
	{Ё}{{\"E}}1
	{Ж}{{\selectfont\char198}}1
	{З}{{\selectfont\char199}}1
	{И}{{\selectfont\char200}}1
	{Й}{{\selectfont\char201}}1
	{К}{{\selectfont\char202}}1
	{Л}{{\selectfont\char203}}1
	{М}{{\selectfont\char204}}1
	{Н}{{\selectfont\char205}}1
	{О}{{\selectfont\char206}}1
	{П}{{\selectfont\char207}}1
	{Р}{{\selectfont\char208}}1
	{С}{{\selectfont\char209}}1
	{Т}{{\selectfont\char210}}1
	{У}{{\selectfont\char211}}1
	{Ф}{{\selectfont\char212}}1
	{Х}{{\selectfont\char213}}1
	{Ц}{{\selectfont\char214}}1
	{Ч}{{\selectfont\char215}}1
	{Ш}{{\selectfont\char216}}1
	{Щ}{{\selectfont\char217}}1
	{Ъ}{{\selectfont\char218}}1
	{Ы}{{\selectfont\char219}}1
	{Ь}{{\selectfont\char220}}1
	{Э}{{\selectfont\char221}}1
	{Ю}{{\selectfont\char222}}1
	{Я}{{\selectfont\char223}}1
	{і}{{\selectfont\char105}}1
	{ї}{{\selectfont\char168}}1
	{є}{{\selectfont\char185}}1
	{ґ}{{\selectfont\char160}}1
	{І}{{\selectfont\char73}}1
	{Ї}{{\selectfont\char136}}1
	{Є}{{\selectfont\char153}}1
	{Ґ}{{\selectfont\char128}}1
	{\{}{{{\color{brackets}\{}}}1 % Цвет скобок {
	{\}}{{{\color{brackets}\}}}}1 % Цвет скобок }
}

\setcounter{tocdepth}{1}

\begin{document}

\begin{titlepage}
\newpage

\

Тут титульник
\end{titlepage}

\newpage
\tableofcontents
\setcounter{page}{2}


\newpage\section{Нейронные сети для задачи классификации} 
	При решении задач классификации необходимо отнести имеющиеся статические образцы (характеристики ситуации на рынке, данные медосмотра, информация о клиенте) к определенным классам. Возможно несколько способов представления данных. Наиболее распространенным является способ, при котором образец представляется вектором. Компоненты этого вектора представляют собой различные характеристики образца, которые влияют на принятие решения о том, к какому классу можно отнести данный образец. Например, для медицинских задач в качестве компонентов этого вектора могут быть данные из медицинской карты больного. Таким образом, на основании некоторой информации о примере, необходимо определить, к какому классу его можно отнести. Классификатор таким образом относит объект к одному из классов в соответствии с определенным разбиением N-мерного пространства, которое называется пространством входов, и размерность этого пространства является количеством компонент вектора.
	
	\vspace{0.5cm}
	Как правило, нейронные сети оказываются наиболее эффективным способом классификации, потому что генерируют фактически большое число регрессионных моделей (которые используются в решении задач классификации статистическими методами).
	
	

	

\newpage\section{Цель лабораторной работы} 
	Цели: 
	\vspace{0.5cm}
	
	Получить практические навыки по работе с искусственными нейронными сетями
	
	\vspace{0.5cm}
	Задачи: 
	
	\vspace{0.5cm}
	1. Разработать и обосновать несколько (не менее 3-х) различных архитектур нейронных сетей прямого распространения (feed-forward) для классификации датасета MNIST.
	
	
\newpage\section{Инструменты} 
	В качестве инструментов для выполнения поставленной цели был выбран язык Python и библиотеки sсikit-learn и Pandas.
	Бибилотека Pandas была использована для подготовки датасета к будущему использованию.
	
	\vspace{0.5cm}
	Библиотека sсikit-learn была использована для построения нейронных сетей. Для этого использовался класс MLPClassifier из sklearn.neural\_network.
	
	\vspace{0.5cm}
	Основные параметры класса MLPClassifier.
	
	\vspace{0.5cm}
	hidden\_layer\_sizes – количество скрытых слоев и нейронов в них. По умолчанию равен (100,).

	
\newpage\section{Эксперименты}
	Для того, чтобы вывести при каких параметрах используемые модели нейронных сетей дадут наиболее точное распознавание, необходимо провести ряд экспериментов с разными архитектурами. 

	\vspace{0.5cm}
	Таблица 1 - Точность прогнозов  при использовании различных топологий многослойного перцептрона.
\begin{longtable}{|p{1cm}|p{9cm}|p{3cm}|}
\hline
№ & Архитектура & Точность (в процентах) \\ 
\hline 
1 & hidden\_layer\_sizes=30, 10, & 95.66 \\
\hline
2 & hidden\_layer\_sizes=25, 10, 10,  & 94.6 \\
\hline 
3 & hidden\_layer\_sizes=30, & 96.19 \\
\hline
\end{longtable}


\newpage\section{Итог}
	...
	
	\vspace{0.5cm}
	Можно предположить, что не стопроцентная точность может быть обусловлена искажениями изображения в некоторых объектах  набора, или же неполноцнным обучением. Так как количество итераций для каждой модели не превышало двухсот. Это было сделано намерено, в целях ускорения эспериментов. Для этого, в модели MLPClassifier был высталвен флаг early\_stopping равный True.
	
	
\end{document}
